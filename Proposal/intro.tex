\chapter{Introduction}
\label{cha:intro}

Since the 1920s, scientists have recognised the existence of galaxies outside the Milky Way \citep{sparke_galaxies_2007}. A century later, we continue to work towards attaining a full picture of the formation and evolution of galaxies. An in-depth analysis of observational and theoretical research into early galactic evolution is given in Chapter \ref{cha:lit-review}, but here I will discuss the motivations and objectives behind my own thesis. Section \ref{sec:motivation} of this chapter will motivate this study, introducing the problem statement and novelty of this proposed work. In particular, I will highlight how \tc, an existing machine-learning model for determining stellar parameters, intersected with the emerging problem of resolving galactic histories from present-day spectra, to motivate this project. Objectives for my thesis will be covered in Section \ref{sec:objectives}, including a breakdown of the main goal of developing a simple training set and stretch goals of increasing complexity.

\section{Motivation}
\label{sec:motivation}

Machine-learning algorithms have been gaining traction in astronomy due to the recent and ongoing flourishing of large-scale surveys like \sdss \citep{york_sloan_2000}, Gaia \citep{gaia_collaboration_gaia_2016}, and \desi \citep{desi_collaboration_early_2024}. The automation of detection, characterisation, and classification processes has streamlined the interpretation of the massive data outputs produced by these missions. When it comes to determining characteristic labels from stellar spectra, \tc \citep{ness_cannon_2015} provides a data-driven, supervised learning approach enabling the discernment of stellar kinematics and chemical composition from a single spectrum. However, an equivalent tool for galaxies does not yet exist.

Recovering galactic properties, including \sfh, chemical composition, and dust composition, is important in not only uncovering a particular galaxy's evolutionary path but also in working towards deriving broader cosmological patterns. While a plethora of studies have used various techniques to retrieve this information from galactic spectra \citep[e.g.][]{mathis_extracting_2006, tojeiro_recovering_2007, silva-lima_optimizing_2025}, the sheer complexity of disentangling features from composite stellar populations and the \ism makes the process slow and challenging. As a result, the effort to determine fundamental galactic parameters for even single galaxies often hinders deeper analysis into large-scale trends and cosmological implications. Thus emerges a need for a machine-learning model, analogous to \tc, which can resolve galaxy parameters from a spectrum.

This intersection between the need for efficient interpretation of galaxy spectra with the availability of algorithms that can be trained to perform such a task is the primary motivator for my project. Ultimately, the aim is to develop a trained algorithm that can return galaxy parameters with high confidence, requiring the input of only a single spectrum. However, there are many intermediate steps to fulfilling this goal, which will be examined in detail in the following section.

\section{Objectives}
\label{sec:objectives}
The overarching objective of this project is to train and test the existing machine-learning software \tc to interpret any given present-day galaxy spectrum for key parameters. Initially, I am working towards resolving \sfh, but other properties like chemical composition and dust content can later be added to increase the complexity of the model. Breaking this down into a series of milestones:
\begin{enumerate}
    \item Construct a library of synthetic galactic spectra using the Python software \textit{FSPS} \citep{conroy_propagation_2009, conroy_propagation_2010}. For this first iteration, all spectra would be generated by varying only \sfh, assuming evolution from the Big Bang to present day ($t_\mathrm{age}\approx13.8$ Gyr).
    \item Train \tc to learn the relationship between the \sfh of a galaxy and its present-day spectrum, using the synthetic spectral library. Investigate how large of a training set is required to achieve a high level of confidence in the model, using $k$-fold cross validation techniques.
    \item Test trained model on a real galaxy spectrum, and assess its \sfh as predicted by \tc against results from another source, for example, a neural network model.
\end{enumerate}
The above objectives can be similarly applied to implementing other variables into the training of \tc. The stretch goal for this project is to also incorporate metallicity as a parameter into the training set. Meaning, that like with \sfh, \tc would also be able to deduce the chemical composition of a galaxy, as well as its stellar populations, from a spectrum. Such capabilities would bring us a step closer to piecing together the complex evolutionary histories of galaxies, and in turn, our universe.