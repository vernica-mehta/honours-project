\chapter{Literature Review}
\label{cha:lit-review}

It is widely accepted that the universe follows the \lcdm model, that is, a spatially flat cosmology comprised of dark energy, cold dark matter, and ordinary (baryonic) matter \citep[see e.g.][for further discussions on cosmological models]{planck_collaboration_planck_2020, lee_shape_2025}. The many complex components of this paradigm pose challenges to understanding the formation and evolution of galaxies, however, significant strides have been made both through observations and simulations in working towards this goal. This chapter contextualises my thesis within the vast array of existing literature concerning the characterisation of galaxies. More specifically, it will unpack observational and numerical contributions to theories of galaxy formation, assessing the relationship between hierarchical star formation and internal and external processes affecting composite stellar populations.

The relevant background information for this thesis can be categorised under two themes: theory and computation. Section \ref{sec:theory} of this chapter will discuss galaxy parametrisation, introducing the concept of labels and relating the formation of individual stellar populations within a galaxy to its broader conditions. This section will also include a brief overview of \textit{FSPS}, a Python library that can be used to put this theory into practice by synthetically generating stellar populations within a galaxy. Section \ref{sec:software} will further focus on studies in computational astronomy which have worked towards resolving galaxy parameters using numerical simulations and data-driven approaches. A deep dive into \tc will be included, to clarify its role as the algorithmic framework for this project. Through this, I will demonstrate the role of my project both in the development of machine-learning technologies for astronomy, and the applications of such technology in further understanding a crucial structural cosmological component.

\section{How to Build a Galaxy}
\label{sec:theory}

Galaxies are shaped by a core set of physical processes --- cosmological accretion, strong stellar-driven winds, black hole feedback, and morphological evolution --- which exchange dialogue with hierarchical bursts of star formation \citep{somerville_physical_2015}.

\subsection{Star Formation Histories}

\subsection{\textit{FSPS: Flexible Stellar Population Synthesis}}

\section{Software for Galaxy Parametrisation}
\label{sec:software}

\subsection{\textit{The Cannon:} A Machine-Learning Algorithm}

\subsection{Neural Networks and Other Modelling}
